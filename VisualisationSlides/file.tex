\documentclass[xcolor=dvipsnames, 12pt]{beamer}

\usepackage[T2A]{fontenc}
\usepackage[utf8]{inputenc}
\usepackage[russian]{babel}
\usepackage{pscyr}
\usepackage{dirtytalk}

% сокращения
\newcommand*\code[1]{\textcolor{RawSienna}{\small \texter{{#1}}}}
\newcommand*\R[1]{{\textcr{\textbf{#1}}}}

\usetheme{metropolis}           % Use metropolis theme
\title{Визуализация данных}

%\date{25 апреля 2018}
%\date{5 июля 2018}
\date{7 февраля 2019}
\author{Е.Н. Матеров}

\institute{ФГБОУ ВО <<Сибирская пожарно-спасательная академия>> ГПС МЧС России \\
кафедра физики, математики и информационных технологий}

\begin{document}
  
  \maketitle
  
\begin{frame}{Наука о данных}

\begin{itemize}

{\small

\item 
% В настоящее время, для решения сложных аналитических задач  одной из самых передовых областей знаний является 
\alert{Наука о данных} (Data Science) --- междисциплинарная область, изучающая проблемы обобщения, анализа, алгоритмизации и представления данных в численной и визуальной формах. 

\vskip0.4cm

\item Наука о данных опирается на инструменты эмпирических наук, статистики, отчетности, анализа, визуализации, бизнес-аналитики, экспертных систем, машинного обучения, баз данных, хранения данных, интеллектуального анализа данных и больших данных. 

}

\end{itemize}

\end{frame}


\begin{frame}{Инструменты для обработки данных}

\begin{figure}[h!]
 \begin{minipage}[h]{0.3\linewidth}
\center{ \includegraphics[scale=0.25]{Excel.png} \\ {\small а)~Microsoft Excel}}
 \end{minipage}
 \hfill
  \begin{minipage}[h]{0.3\linewidth}
\center{  \includegraphics[scale=0.25]{STATISTICA.png} \\ {\small б)~Statistica}}
 \end{minipage}
 \hfill
  \begin{minipage}[h]{0.3\linewidth}
\center{  \includegraphics[scale=0.14]{SPSS.jpg} \\ {\small в)~IBM SPSS}}
 \end{minipage} 
 %\caption{Визуализация данных}
 %\label{fig:ClevelandMcG1}
\end{figure}

\begin{figure}[h!]
 \begin{minipage}[h]{0.3\linewidth}
\center{ \includegraphics[scale=0.13]{Matlab.png} \\ {\small г)~Matlab}}
 \end{minipage}
 \hfill
  \begin{minipage}[h]{0.3\linewidth}
\center{  \includegraphics[scale=0.075]{Python.png} \\ {\small д)~Python}}
 \end{minipage}
 \hfill
  \begin{minipage}[h]{0.3\linewidth}
\center{  \includegraphics[scale=0.076]{R_logo.png} \\ {\small е)~R}}
 \end{minipage} 
 %\caption{Визуализация данных}
 %\label{fig:ClevelandMcG2}
\end{figure}

\end{frame}

\begin{frame}{Преимущества языков программирования \\для анализа данных}

\begin{itemize}
\item \alert{Бесплатность}

\vskip0.2cm

\item Возможность обработки больших объёмов данных

\vskip0.2cm

\item Автоматизация вычислений, воспроизводимость результата

\vskip0.2cm

\item Использование современных актуальных алгоритмов

\vskip0.2cm

\item Графические возможности

\vskip0.2cm

\item Организация проектов, работа с Git

\vskip0.2cm

\item Формирование интерактивных отчётов

\end{itemize}


\end{frame}  
  
\begin{frame}{Анализ данных}

\vspace*{-1.2cm}
\begin{figure}[h!]
\hspace*{-1.2cm}
 \centering
 \includegraphics[scale=0.55]{data-science.pdf}
\end{figure}

\vspace*{-1.2cm}
\begin{thebibliography}{99}
\bibitem{book:R4DS}
{\scriptsize
\textbf{Hadley Wickham, Garrett Grolemund}

R for Data Science: Import, Tidy, Transform, Visualize, and Model Data. \\ O'Reilly, 2016. 

}
\end{thebibliography}

\end{frame}  

\begin{frame}{}

%\vspace*{-1.2cm}
\begin{figure}[h!]
\hspace*{-1.05cm}
 \centering
 \includegraphics[scale=0.235]{posob.png}
\end{figure}

\end{frame}  
  
\begin{frame}{Визуализация данных}

\vspace*{-1.2cm}
\begin{figure}[h!]
\hspace*{-1.2cm}
 \centering
 \includegraphics[scale=0.55]{visualization.pdf}
\end{figure}

\vspace*{-1.2cm}
{\footnotesize
\say{
Самая большая ценность графика --- это когда он заставляет нас замечать то, чего мы никогда не ожидали увидеть!
}

\hspace*{4.5cm}Джон Тьюки (американский статистик)
}

\end{frame}  



%\begin{frame}{Визуализация данных в R как отображение}

%\begin{figure}[h!]
%\hspace*{-1cm}
% \centering
% \includegraphics[scale=0.45]{R4DS.JPG}

%\end{figure}

%\end{frame}

%\section{Виды графиков \newline {\large Гистограмма}}

%\begin{frame}{Гистограмма}

%\begin{figure}[h!]
%\hspace*{-1cm}
% \centering
% \includegraphics[scale=0.56]{hist_fire.pdf}

%\end{figure}

%\end{frame}

%\section{Виды графиков \newline {\large График плотности}}

%\begin{frame}{График плотности}

%\begin{figure}[h!]
%\hspace*{-1cm}
% \centering
% \includegraphics[scale=0.56]{dens_fire.pdf}

%\end{figure}

%\end{frame}

%\section{Виды графиков \newline {\large Столбиковая диаграмма}}

%\begin{frame}{Столбиковая диаграмма}

%\begin{figure}[h!]
%\vspace*{-0.5cm}
%\hspace*{-1cm}
% \centering
% \includegraphics[scale=0.42]{Stolbik1.pdf}

%\end{figure}

%\end{frame}
  
    
  \begin{frame}{Характеристика восприятия (в порядке убывания)}
  
\begin{enumerate}%[topsep=2.5pt,noitemsep,itemsep=2pt]

\item Положение относительно общего масштаба

\vskip0.2cm

\item Положение относительно несимметричной шкалы

\vskip0.2cm

\item Длина, направление, угол

\vskip0.2cm

\item Площадь

\vskip0.2cm

\item Объем, кривизна

\vskip0.2cm

\item Затенение, насыщение цветом

\end{enumerate}

\begin{thebibliography}{99}
\bibitem{paper:Cleveland}
{\scriptsize
\textbf{William S. Cleveland, Robert McGill}

Graphical Preception: Theory, Experimentation, and Application to the Development of Graphical Methods. Journal of the American Statistical Association, 1984. Vol.~79, No.~387, P.~531--534. 

}

\end{thebibliography}
  
\end{frame}
  
\begin{frame}

\begin{figure}[h!]
 \begin{minipage}[h]{0.3\linewidth}
\center{ \includegraphics[scale=0.15]{skitch1.png} \\ {\small а)~положение}}
 \end{minipage}
 \hfill
  \begin{minipage}[h]{0.3\linewidth}
\center{  \includegraphics[scale=0.15]{skitch2.png} \\ {\small б)~длина}}
 \end{minipage}
 \hfill
  \begin{minipage}[h]{0.3\linewidth}
\center{  \includegraphics[scale=0.15]{skitch3.png} \\ {\small в)~угол}}
 \end{minipage} 
 %\caption{Визуализация данных}
 %\label{fig:ClevelandMcG1}
\end{figure}

\begin{figure}[h!]
 \begin{minipage}[h]{0.3\linewidth}
\center{ \includegraphics[scale=0.15]{skitch4.png} \\ {\small г)~направление}}
 \end{minipage}
 \hfill
  \begin{minipage}[h]{0.3\linewidth}
\center{  \includegraphics[scale=0.15]{skitch5.png} \\ {\small д)~форма}}
 \end{minipage}
 \hfill
  \begin{minipage}[h]{0.3\linewidth}
\center{  \includegraphics[scale=0.15]{skitch6.png} \\ {\small е)~площадь}}
 \end{minipage} 
 %\caption{Визуализация данных}
 %\label{fig:ClevelandMcG2}
\end{figure}

\end{frame}

\begin{frame}{Общие рекомендации по визуализации}

\begin{enumerate}

\item Используйте правильный \alert{тип графика}

\vskip0.2cm

\item Располагайте данные в \alert{порядке}, согласно иерархии

\vskip0.2cm

\item Используйте \alert{панелирование}, цвет, форму и площадь для упрощения

\vskip0.2cm

\item У графика должно быть \alert{название} и полная \alert{легенда}

\vskip0.2cm

\item Используйте правильный выбор \alert{масштаба}

\vskip0.2cm

\item Используйте максимально \alert{простой} дизайн графиков

\end{enumerate}

\end{frame}

\begin{frame}{Классификация типов данных}

\begin{enumerate}

\item \alert{Количественные} (дискретные и непрерывные) данные

\vskip0.2cm

\item \alert{Категориальные} (номинальные и порядковые) данные

\vskip0.2cm

\item \alert{Хронологические} данные

\vskip0.2cm

\item \alert{Текстовые} данные

\vskip0.2cm

\item \alert{Пространственные} (географические) данные

\end{enumerate}

\end{frame}

\section{Основные виды графиков \newline {\large классификация по типам данных}}

\begin{frame}

\begin{center}
  \begin{tabular}{ l l l }
    \hline
    переменная ($x$) & отклик ($y$) & \phantom{1}тип графика \\ \hline
    численная &  & \begin{tabular}{l} гистограмма \\ график плотности \end{tabular} \\  \\
    категорная & & \phantom{1}столбиковая диаграмма \\ \\
    численная & численный & \phantom{1}диаграмма рассеяния \\ \\
    хронологическая & численный & \phantom{1}линейный график \\ \\
    категорная & численный & \phantom{1}диаграмма размаха \\ \\
    категорная & категорный & \phantom{1}мозаичная диаграмма \\
    \hline
  \end{tabular}
\end{center}

\end{frame}

\begin{frame}{Визуализация данных в R как отображение}

\begin{figure}[h!]
\hspace*{-1cm}
 \centering
 \includegraphics[scale=0.45]{R4DS.JPG}

\end{figure}

\end{frame}

\section{Виды графиков \newline {\large Гистограмма}}

\begin{frame}

\begin{figure}[h!]
\hspace*{-1cm}
 \centering
 \includegraphics[scale=0.6]{hist_fire.pdf}

\end{figure}

\end{frame}

\section{Виды графиков \newline {\large График плотности}}

\begin{frame}

\begin{figure}[h!]
\hspace*{-1cm}
 \centering
 \includegraphics[scale=0.6]{dens_fire.pdf}

\end{figure}

\end{frame}

\section{Виды графиков \newline {\large Столбиковая диаграмма}}

\begin{frame}

\begin{figure}[h!]
\hspace*{-1cm}
 \centering
 \includegraphics[scale=0.42]{Stolbik.pdf}

\end{figure}

\end{frame}

\begin{frame}

\begin{figure}[h!]
\hspace*{-1cm}
 \centering
 \includegraphics[scale=0.42]{Stolbik3.pdf}

\end{figure}

\end{frame}

\begin{frame}

\begin{figure}[h!]
\hspace*{-1cm}
 \centering
 \includegraphics[scale=0.42]{Stolbik2.pdf}

\end{figure}

\end{frame}

\section{Виды графиков \newline {\large Диаграмма Кливленда}}

\begin{frame}

\begin{figure}[h!]
\hspace*{-1cm}
 \centering
 \includegraphics[scale=0.42]{question18_perc.pdf}

\end{figure}

\end{frame}

\section{Виды графиков \newline {\large Диаграмма рассеяния}}

\begin{frame}

\begin{figure}[h!]
\hspace*{-1cm}
 \centering
 \includegraphics[scale=0.44]{Scatter_plot1.pdf}

\end{figure}

\end{frame}

\begin{frame}

\begin{figure}[h!]
\hspace*{-1cm}
 \centering
 \includegraphics[scale=0.43]{Scatter_plot2.pdf}

\end{figure}

\end{frame}

\section{Виды графиков \newline {\large Линейный график}}

\begin{frame}

\begin{figure}[h!]
\hspace*{-1cm}
 \centering
 \includegraphics[scale=0.42]{example_lineplot1.pdf}

\end{figure}

\end{frame}

\begin{frame}

\begin{figure}[h!]
\hspace*{-1cm}
 \centering
 \includegraphics[scale=0.44]{example_areaplot.pdf}

\end{figure}

\end{frame}

\begin{frame}

\begin{figure}[h!]
\hspace*{-1cm}
 \centering
 \includegraphics[scale=0.38]{panel_grey.pdf}

\end{figure}

\end{frame}

\begin{frame}

\begin{figure}[h!]
\hspace*{-1cm}
 \centering
 \includegraphics[scale=0.55]{smooth_temp.pdf}

\end{figure}

\end{frame}

\begin{frame}

\begin{figure}[h!]
\hspace*{-1cm}
 \centering
 \includegraphics[scale=0.5]{example_temp.pdf}

\end{figure}

\end{frame}

\begin{frame} 

\begin{figure}[h!]
\hspace*{-1cm}
 \centering
 \includegraphics[scale=0.45]{diff_temp.pdf}

\end{figure}

\end{frame}


%\begin{frame}

%\begin{figure}[h!]
%\hspace*{-1cm}
% \centering
% \includegraphics[scale=0.37]{panel_grey.pdf}

%\end{figure}

%\end{frame}

\section{Виды графиков \newline {\large Диаграмма размаха}}

\begin{frame}

\begin{figure}[h!]
\hspace*{-0.9cm}
 \centering
 \includegraphics[scale=0.19]{boxplot.png}

\end{figure}

\end{frame}

\begin{frame}

\begin{figure}[h!]
\hspace*{-1cm}
 \centering
 \includegraphics[scale=0.4]{boxplot_question19.pdf}

\end{figure}

\end{frame}

\begin{frame}

\begin{figure}[h!]
\hspace*{-1cm}
 \centering
 \includegraphics[scale=0.43]{boxplot1b.pdf}

\end{figure}

\end{frame}

\section{Виды графиков \newline {\large Мозаичная диаграмма}}

\begin{frame}

\begin{figure}[h!]
\hspace*{-1cm}
 \centering
 \includegraphics[scale=0.5]{mosaic1e.pdf}

\end{figure}

\end{frame}

\begin{frame}

\begin{figure}[h!]
\hspace*{-1cm}
 \centering
 \includegraphics[scale=0.37]{mosaic_new.pdf}

\end{figure}

\end{frame}

\begin{frame}

\begin{figure}[h!]
\hspace*{-1cm}
 \centering
 \includegraphics[scale=0.4]{calend_mater.pdf}

\end{figure}

\end{frame}



\section{Представление одних и тех же данных различными способами}

\begin{frame}

\begin{figure}[h!]
\hspace*{-1cm}
 \centering
 \includegraphics[scale=0.4]{Died.pdf}

\end{figure}

\end{frame}

\begin{frame}

\begin{figure}[h!]
\hspace*{-1cm}
 \centering
 \includegraphics[scale=0.4]{Dumbell_died.pdf}

\end{figure}

\end{frame}

\begin{frame}

\begin{figure}[h!]
\hspace*{-1cm}
 \centering
 \includegraphics[scale=0.36]{Demography_pyr.pdf}

\end{figure}

\end{frame}

\begin{frame}

\begin{figure}[h!]
\hspace*{-1cm}
 \centering
 \includegraphics[scale=0.155]{Demography.png}

\end{figure}

\end{frame}

\section{Специальные виды графиков \newline {\large Тепловые диаграммы}}

\begin{frame}

\begin{figure}[h!]
\hspace*{-1cm}
 \centering
% \includegraphics[scale=0.325]{Fire_calendar.pdf}
 \includegraphics[scale=0.35]{Fire_compare.pdf}

\end{figure}

\end{frame}

\section{Специальные виды графиков \newline {\large Риджлайны}}

\begin{frame}

\begin{figure}[h!]
\hspace*{-1cm}
 \centering
 \includegraphics[scale=0.47]{example_ridges1.pdf}

\end{figure}

\end{frame}

\section{Специальные виды графиков \newline {\large Облака слов}}

\begin{frame}

\begin{figure}[h!]
\hspace*{-1cm}
 \centering
 \includegraphics[scale=0.38]{word_cloud.pdf}

\end{figure}

\end{frame}

\section{Специальные виды графиков \newline {\large Картограмма}}

\begin{frame}

\begin{center}
\begin{figure}[h!]
\hspace*{-1cm}
 \centering
 \includegraphics[scale=0.25]{plot_zoom_png.png}
\end{figure}
\end{center}

\end{frame}

\begin{frame}[standout]
  \textbf{Спасибо за внимание!}
\end{frame}
  
\end{document}